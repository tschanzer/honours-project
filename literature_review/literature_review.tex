% imports
\documentclass[titlepage]{article}
\usepackage[utf8]{inputenc}
\usepackage[a4paper, total={16cm, 23cm}, top=3.5cm]{geometry}
\usepackage[dvipsnames]{xcolor}
\usepackage{
    amsmath,
    amssymb,
    amsthm,
    fancyhdr,
    siunitx,
    bm,
    lipsum,
    standalone,
    tikz,
    booktabs,
    enumitem,
    array,
}
\usepackage[colorlinks=true, allcolors=linkcolor]{hyperref}
\usepackage[nameinlink]{cleveref}
\usepackage[
    backend=biber,
    bibstyle=ext-authoryear,
    citestyle=ext-authoryear-comp,
    sorting=nyt,
    uniquename=false,
    maxbibnames=99,
    giveninits=true,
]{biblatex}

% bibliographic options
\DeclareFieldFormat[article]{volume}{\mkbibbold{#1}}
\DeclareFieldFormat[article]{number}{\mkbibparens{#1}}
\DeclareFieldFormat[article]{pages}{#1}
\renewcommand*{\volnumdelim}{}
\renewbibmacro{in:}{}
\addbibresource{references.bib}

% ref options
\crefname{section}{\S}{\S\S}
\crefname{subsection}{\S}{\S\S}
\crefname{equation}{}{}
\newcommand\crefrangeconjunction{--}
\def\equationautorefname~#1\null{(#1)\null}
\numberwithin{equation}{section}
\definecolor{linkcolor}{RGB}{51, 54, 142}

\makeatletter
\patchcmd{\math@cr@@@align}{\cr}{\global\let\df@label\@empty\cr}{}{}
\makeatother

% page style options
\pagestyle{fancy}
\fancyhead{}
\rhead{Title}
\renewcommand{\headrulewidth}{0.5pt}
\setlength{\headheight}{15pt}
\setlength\parindent{0pt}
\setlength\parskip{6pt}

% math macros
\renewcommand{\d}[1]{\mathrm{d}#1}
\newcommand{\diff}[2]{\frac{\mathrm{d} #1}{\mathrm{d} #2}}
\newcommand{\ddiff}[2]{\frac{\mathrm{d}^2 #1}{\mathrm{d} {#2}^2}}
\newcommand{\pdiff}[2]{\frac{\partial #1}{\partial #2}}
\renewcommand\vec{\bm}
\newcommand{\uvec}[1]{\vec{\hat{#1}}}
\newcommand{\grad}{\vec{\nabla}}
\newcommand{\prandtl}{\ensuremath{\mathrm{Pr}}}
\newcommand{\rayleigh}{\ensuremath{\mathrm{Ra}}}

% text macros
\newcommand{\rb}{Rayleigh-B\'{e}nard}

\begin{document}
\begin{titlepage}
\vfill~

\begin{center}
    {\Huge \textbf{%
        Title
    }} \\
    \vspace{0.75cm}
    {\Large\textbf{Honours Literature Review}} \\
    \vspace{0.75cm}
    {\Large\textbf{Thomas D. Schanzer}} \\
    \vspace{6pt}
    {\large Supervisor: Prof. Steven Sherwood} \\
    \vspace{0.75cm}
    {\large%
        School of Physics

        Climate Change Research Centre and
        ARC Centre of Excellence for Climate Extremes

        University of New South Wales, Sydney, Australia
    }
\end{center}
\vfill
\begin{center}
{\large\textbf{Abstract}}

\begin{minipage}{13cm}
    Text
\end{minipage}
\end{center}
\vfill
\end{titlepage}

\newpage
\tableofcontents

\newpage
\pagestyle{fancy}
\thispagestyle{fancy}

\section{Introduction}
\subsection{Traditional parametrisation schemes}

\newpage
\section{Theory of parametrisation}

\newpage
\section{Data-driven parametrisation schemes}

\newpage
\section{Parametrisation development using toy models}

\newpage
\section{\rb{} convection and numerical methods}
\subsection{Problem statement}
\rb{} convection is the motion of a fluid confined between two horizontal
isothermal plates, the temperature of the bottom plate being higher than that
of the top plate.
% TODO: brief description of history and phenomenology

The governing equations for the flow are derived from the Navier-Stokes
equations of mass, energy and momentum conservation (see, e.g.,
\textcite{chandrasekhar1961}). The density, $\rho$, of the fluid is related to
its temperature $T$ by the linear equation of state
\[
    \rho = \rho_0 [1 - \alpha(T - T_0)],
\]
where $\alpha$ is the (constant) volume coefficient of thermal expansion and
$\rho_0$ and $T_0$ are the base-state density and temperature such that $\rho =
\rho_0$ when $T = T_0$. The key assumption is that density variations are small
($\alpha (T - T_0) \ll 1$), which allows the governing equations to be
simplified under the \emph{Boussinesq approximation}. The Boussinesq
approximation involves first writing the pressure, $p$, of the fluid as
\[
    p = p_0 - \rho_0 gz + p',
\]
where $p_0$ is an arbitrary constant, $g$ is the acceleration due to gravity
and $z$ is the vertical coordinate. $p'$ is the (time-varying) deviation from
a hydrostatically balanced background profile $p_0 - \rho_0 gz$
in which the upward pressure gradient force per unit volume $\rho_0 g$ cancels
the downward weight force per unit volume $-\rho_0 g$. Since
$\alpha (T - T_0) \ll 1$, density variations are neglected everywhere except
in their contribution to the weight force, leading to a net buoyant
(background pressure gradient plus weight) force per unit mass
\[
    \frac{\rho_0 - \rho}{\rho_0} g = \alpha (T - T_0) g.
\]

I adopt the governing equations as they are derived by
\textcite{chandrasekhar1961}:
\begin{alignat}{2}
    \label{eqn:dim_momentum}
    \pdiff{\vec{u}}{t} + \vec{u} \cdot \grad \vec{u}
        &= -\frac{1}{\rho_0} \grad p' + \alpha (T - T_0) g \uvec{z}
        + \nu \nabla^2 \vec{u}
    &\quad& \text{(momentum conservation),} \\
    \label{eqn:dim_energy}
    \pdiff{T}{t} + \vec{u} \cdot \grad T
        &= \kappa \nabla^2 T,
    && \text{(energy conservation), and} \\
    \label{eqn:dim_incompressible}
    \grad \cdot \vec{u} &= 0
    && \text{(incompressibility).}
\end{alignat}
$\vec{u}$ is the fluid velocity, $t$ is time, $\uvec{z}$ is the upward
unit vector, $\nu$ is the (constant) kinematic viscosity and $\kappa$ is the
thermal diffusivity (also constant).

The parametrisation test-bed developed in this work solves the governing
equations in a two-dimensional domain $(x,z) \in [0, d] \times [0, L]$, subject
to no-slip, isothermal boundary conditions on the top and bottom plates,
\begin{alignat}{3}
    \label{eqn:dim_bc_bot}
    \vec{u} &= \vec{0}, &\quad T &= T_0 + \frac{\delta T}{2}
    &\qquad& \text{at } z = 0 \text{ and} \\
    \label{eqn:dim_bc_top}
    \vec{u} &= \vec{0}, &\quad T &= T_0 - \frac{\delta T}{2}
    &\qquad& \text{at } z = d,
\end{alignat}
and periodic boundary conditions in the horizontal,
\begin{alignat}{2}
    \label{eqn:dim_bc_sides}
    \vec{u}(x=0) &= \vec{u}(x=L) &\quad \text{and} \quad T(x=0) &= T(x=L).
\end{alignat}
$\delta T$ is the constant temperature difference between the plates.

\subsection{Nondimensionalisation and scale analysis}
It is helpful to nondimensionalise the governing equations
\crefrange{eqn:dim_momentum}{eqn:dim_bc_sides}; this is not only useful
for numerical work but also gives insight into the different flow regimes
that are possible. A range of nondimensionalisations are used in fluid
dynamics literature; I adopt a common one which is used, for example, by
\textcite{ouertatani2008,stevens2010} and is suitable for the turbulent
convective regime.

For low-viscosity, turbulent flow, a suitable time scale is the
\emph{free-fall time} $t_0$, which is the time for a fluid element with
constant temperature $T = T_0 - \delta T$ to fall from the top plate
to the bottom plate under the influence of buoyancy alone. It is simple
to show that
\[
    t_0 \sim \left( \frac{d}{g \alpha \delta T} \right)^{1/2},
\]
ignoring a factor of $\sqrt{2}$. The obvious length and temperature
scales are the plate separation $d$ and temperature difference $\delta T$,
respectively.

Making the substitutions $p'/\rho_0 \to \pi$ and $T - T_0 \to \theta$
in \crefrange{eqn:dim_momentum}{eqn:dim_bc_sides} and expressing all
variables in units of $t_0$, $d$ and $\delta T$ leads to the dimensionless
equations
\begin{align}
    \label{eqn:momentum}
    \pdiff{\vec{u}}{t} + \vec{u} \cdot \grad \vec{u}
        &= -\grad \pi + \left( \frac{\prandtl}{\rayleigh}\right)^{1/2}
        \nabla^2 \vec{u} + \theta \uvec{z}, \\
    \label{eqn:energy}
    \pdiff{\theta}{t} + \vec{u} \cdot \grad \theta
        &= (\rayleigh\,\prandtl)^{1/2} \, \nabla^2 \theta, \quad \text{and} \\
    \label{eqn:incompressible}
    \grad \cdot \vec{u} &= 0,
\end{align}
with boundary conditions
\begin{gather}
\begin{alignat}{3}
    \label{eqn:bc_bot}
    \vec{u} &= \vec{0}, &\quad \theta &= +\frac{1}{2}
    &\qquad& \text{at } z = 0, \\
    \label{eqn:bc_top}
    \vec{u} &= \vec{0}, &\quad \theta &= -\frac{1}{2}
    &\qquad& \text{at } z = d,
\end{alignat} \\
\begin{alignat}{2}
    \label{eqn:bc_sides}
    \vec{u}(x=0) &= \vec{u}(x=L)
    &\quad \text{and} \quad \theta(x=0) &= \theta(x=L).
\end{alignat}
\end{gather}
There are only two parameters, both dimensionless: the \emph{Prandtl number}
\[
    \prandtl \equiv \frac{\nu}{\kappa},
\]
and the \emph{Rayleigh number}
\[
    \rayleigh \equiv \frac{g \alpha d^3 \delta T}{\kappa \nu}.
\]
The Rayleigh number can be interpreted as the ratio of the time scale
for thermal transport by convection to the the time scale for thermal
transport by conduction. Detailed theoretical analysis of the governing
equations (see, e.g., \textcite{chandrasekhar1961} and the seminal work
by \textcite{rayleigh1916}) reveals that there exists a critical
Rayleigh number, below which the equations have a stable equilibrium
with the fluid at rest and a linear conductive temperature profile.
Above the critical value, the equilibrium is unstable and small
perturbations lead to the formation of a regular series of rotating
convection cells.

\newpage
\section{Conclusion}

\emergencystretch=5em
\printbibliography
\end{document}
