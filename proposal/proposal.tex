\documentclass[titlepage]{article}
\usepackage[utf8]{inputenc}
\usepackage[a4paper, total={16cm, 23cm}, top=3.5cm]{geometry}
\usepackage{
    amsmath,
    amssymb,
    amsthm,
    fancyhdr,
    siunitx,
    bm,
    lipsum,
}
\usepackage[dvipsnames]{xcolor}
\definecolor{linkcolor}{RGB}{51, 54, 142}
\usepackage[colorlinks=true, allcolors=linkcolor]{hyperref}
\usepackage[
    backend=biber,
    bibstyle=ext-authoryear,
    citestyle=ext-authoryear-comp,
    sorting=nyt,
    uniquename=false,
    maxbibnames=99,
    giveninits=true,
]{biblatex}
\DeclareFieldFormat[article]{volume}{\mkbibbold{#1}}
\DeclareFieldFormat[article]{number}{\mkbibparens{#1}}
\DeclareFieldFormat[article]{pages}{#1}
\renewcommand*{\volnumdelim}{}
\renewbibmacro{in:}{}
\addbibresource{references.bib}

\pagestyle{fancy}
\fancyhead{}
\rhead{Research Proposal}
\renewcommand{\headrulewidth}{0.5pt}

\setlength{\headheight}{15pt}
\setlength\parindent{0pt}
\setlength\parskip{6pt}

\renewcommand{\d}[1]{\mathrm{d}#1}
\newcommand{\diff}[2]{\frac{\mathrm{d} #1}{\mathrm{d} #2}}
\newcommand{\ddiff}[2]{\frac{\mathrm{d}^2 #1}{\mathrm{d} {#2}^2}}
\newcommand{\pdiff}[2]{\frac{\partial #1}{\partial #2}}
\renewcommand\vec{\bm}
\newcommand{\uvec}[1]{\vec{\hat{#1}}}

\def\equationautorefname~#1\null{%
    (#1)\null
}

\begin{document}
\begin{titlepage}
\begin{center}
    {\Huge \textbf{%
        Title
    }} \\
    \vspace{0.75cm}
    {\Large\textbf{Honours Research Project Proposal}} \\
    \vspace{0.75cm}
    {\Large\textbf{Thomas D. Schanzer}} \\
    \vspace{6pt}
    {\large Supervisor: Prof. Steven Sherwood} \\
    \vspace{0.75cm}
    {\large%
        School of Physics and \\
        Climate Change Research Centre,\\
        ARC Centre of Excellence for Climate Extremes

        University of New South Wales, Sydney, Australia
    }
\end{center}
\vspace{1cm}
\begin{center}
{\large\textbf{Abstract}}

\begin{minipage}{13cm}
    \lipsum[1]
\end{minipage}
\end{center}
\vspace{1cm}
\tableofcontents
\end{titlepage}

\newpage
\pagestyle{fancy}
\thispagestyle{fancy}
\section{Introduction}
% how do earth system models work? what do they do? why do we need them?
The Earth system (including atmosphere, ocean and land) is distinguished both
in its complexity and its influence on all terrestrial life. If the wellbeing
of humanity in particular is to be preserved, it is difficult to overstate the
importance of understanding and predicting this system's behaviour---both
short-term weather and long-term climate. This understanding and predictive
skill informs important decisions and government policies that have the
potential to reduce our vulnerability to extreme events (e.g., floods,
droughts, fires, tropical cyclones) and long-term climate change, and lessen
negative human impacts (e.g., greenhouse gas emissions) to sustainable levels.

A significant part of our understanding and predictive skill is derived from
numerical modelling of the Earth system. Like many other models, these aim
to predict the time evolution of an initial state (e.g., pressure, temperature,
wind velocity in the atmosphere) given a set of boundary conditions and
external forcings.
% what makes modelling hard? why do we need parametrisation?
Unfortunately, there are many obstacles to accurate modelling. Arguably the
most fundamental of these is chaos: even with a perfect model and unlimited
computing resources, arbitrarily small differences in initial conditions
grow exponentially. This constrains short-term predictability. Second,
the Earth system comprises a vast number of interacting components, such as
water in all three phases, solar radiation and clouds (just to name a few which
are relevant to atmospheric modelling). In other words, the system is
high-dimensional. Third, the dynamics occur on a wide spectrum of spatial
and temporal scales. These range from large-scale, slowly-evolving motions
such as ocean gyres and the atmospheric Hadley circulation, to mid-scale,
transient weather systems (termed synoptic-scale) and ocean eddies, to
small-scale wind gusts, tornadoes, water waves and turbulence. All scales
and their cross-interactions influence the overall dynamics.

Atmosphere and ocean models solve the differential equations that govern fluid
flow with a finite spatial and temporal resolution and are therefore only able
to resolve behaviour whose scale is of the same order of magnitude as the
resolution or larger. The achievable resolution is constrained by the
capabilities, availability and cost of modern computing resources. A typical
atmospheric global climate model has a spatial resolution on the order of
$\ang{1}$ latitude/longitude, roughly corresponding to the size of the entire
Greater Sydney area from Katoomba to Bondi. Sub-grid scale features, such as
invidual clouds, cannot be explicitly resolved, but to ignore them completely
would introduce unacceptable biases (inaccuracies, relative to observations)
in the model output. The same is true for the numerous components of the
Earth system that influence the fluid dynamics but are not directly predicted
by the fluid equations, such as solar radiation and land interactions (e.g.,
moisture and heat fluxes from vegetation, soil and water bodies).

% how has parametrisation been done historically?
The process of using the information available in the model to estimate the
effect of these unresolved processes on the coarse-scale variables is known as
\emph{parametrisation}, which will be the topic of the thesis. Traditional
parametrisations are often based on heavily simplified conceptual models of the
processes in question. For example, the Community Atmosphere Model
\parencite{cam5} parametrises atmospheric convection by considering an
ensemble of rising updraft plumes, making assumptions about their mass flux
and initiation conditions \parencite{zhang1995}. It is now known that these
methods may under-predict the variance and extreme values of their outputs
(e.g., precipitation). % citation needed! see Shamekh et al.
This is detrimental to predictions of future climate and climate extremes.

% what are the limitations of traditional methods?
% what work has been done to resolve these issues?
% how much success has there been?
% what problems remain to be solved?

% narrow it down: what problem will we solve?
% brief review of relevant literature - results, strengths, weaknesses

\section{Proposed methods}

% Possible issues

\section{Preliminary results}

\end{document}
